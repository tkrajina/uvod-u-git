\chapter*{Instalacija, konfiguracija i prvi projekt}
\addcontentsline{toc}{chapter}{Instalacija, konfiguracija i prvi projekt}

\begin{itemize}
   \item .git
   \item .gitignore
   \item komande
   \item Username, password
   \item boje
   \item ostale konfiguracije
   \item lokalna i globalna konfiguracija
\end{itemize}

\section*{Instalacija}
\addcontentsline{toc}{section}{Instalacija}

Instalacija gita je relativno jednostavna. Ukoliko ste na na nekom od linuksoidnih operativnih sustava -- sigurno postoji paket za instalaciju gita. 
Za sve ostale, postoje jednostavne instalacije, a svi linkovi su dostupni na službenim web stranicama\footnote{http://git-scm.com/download}.

Važno je napomenuti da su to samo \emph{osnovni paketi}. 
Oni će biti dovoljni za primjere koji slijede. 
No, za mnoge specifične scenarije postoje dodaci s kojima se git naredbe "obogaćuju" i omogućavaju nove stvari.

\section*{Prvi git repozitorij}
\addcontentsline{toc}{section}{Prvi git repozitorij}

Ukoliko ste naviknuti na TFS, subversion ili CVS onda si vjerojatno zamišljate da je za ovaj korak potrebno neko računalo na kojem je instaliran poseban servis ("daemon") i kojemu je potrebno nekako dati do znanja da želite imati novi repozitorij na njemu.
Vjerojatno mislite i to da je sljedeći korak nekako preuzeti taj projekt s tog udaljenog računala/servisa.

S gitom je jednostavnije. 
\emph{Apsolutno svaki direktorij može postati git repozitorij.}
Ne mora \emph{uopće} postojati udaljeni server i neki centralni repozitorij kojeg koriste (i) ostali koji rade na projektu.
Ako Vam je to neobično, stvar je još čudnija -- ako već postoji udaljeni repozitorij s kojeg preuzimate izmjene od drugih programera on ne mora biti jedan jedini.
\emph{Mogu postojati deseci takvih udaljenih repozitorija, sami ćete odlučiti na koje ćete "slati" svoje izmjene i s kojih preuzimati izmjene.}

No, idemo sad na prvi i najjednostavniji korak -- stvoriti ćemo novi direktorij \verb+moj-prvi-projekt+ i stvoriti novi repozitorij u njemu:

\begin{verbatim}
$ mkdir moj-prvi-projekt
$ cd moj-prvi-projekt
$ git init
Initialized empty Git repository in /home/user/moj-prvi-projekt/.git/
$ 
\end{verbatim}

I to je to. 

Ukoliko idete pogledati kakva se to čarolija desila u s tim \verb+git init+, otkriti ćete da je stvoren direktorij \verb+.git+.
U principu, cijela povijest, sve grane, čvorovi i komentari, apsolutno sve vezano uz repozitorij se čuva u tom direktoriju.
Zatreba vam ikad sigurnosna kopija cijelog repozitorija -- sve što trebate napraviti je da sve lokalne promjene spremite (\emph{commitate}) u git i spremite negdje arhivu s tim \verb+.git+ direktorijem.

\section*{Git naredbe}
\addcontentsline{toc}{section}{Git naredbe}

U prethodnom primjeru smo u našem direktoriju inicijalizirali git repozitorij s naredbom \verb+git ini+.
Općenito, git naredbe uvijek imaju sljedeći format:

\begin{verbatim}
git <komanda> <opcija1> <opcija2> ...
\end{verbatim}

Izuzetak je pomoćni grafički program s kojim se može pregledavati povijest projekta, a koji dolazi u instalaciji s gitom -- \verb+gitk+.

Za svaku git naredbu možete dobiti \emph{help} s:

\begin{verbatim}
git help <komanda>
\end{verbatim}

Na primjer, \verb+git help init+ ili \verb+git help config+.

\section*{Osnovna konfiguracija}
\addcontentsline{toc}{section}{Osnovna konfiguracija}

Ima nekoliko osnovnih stvari koje \emph{morate} konfigurirati da bi ste nastavili normalan rad. 
Sva git konfiguracija se postavlja pomoću naredbe \verb+git init+. 
Postavke mogu biti \emph{lokalne} (odnosno vezane uz jedan jedini projekt) ili \emph{globalne} (vezane uz korisnika na računalu).

Globalne postavke se postavljaju s:

\begin{verbatim}
git config --global <naziv> <vrijednost>
\end{verbatim}

\dots{}i one se spremaju u datoteku \verb+.gitconfig+ u vašem \emph{home} direktoriju.

Lokalne postavke se spremaju u \verb+.git+ direktorij u direktoriju koji sadrži vaš repozitorij.

Za normalan rad na nekom projektu, drugi korisnici trebaju znati tko je točno radio izmjene na kodu (\emph{commit}ove), zato trebate postaviti ime i prezime i email adresu:

\begin{verbatim}
$ git config --global user.name "Ana Anić"
$ git config --global user.email "ana.anic@privatna.domena.com"
\end{verbatim}

Imate li neki repozitorij koji je vezan za posao, i možda ne želite da se identificirate sa svojom privatnom domenom, tada \emph{u tom direktoriju} trebate postaviti drukčije postavke:

\begin{verbatim}
$ git config user.name "Ana Anić"
$ git config user.email "ana.anic@poslodavac.hr"
\end{verbatim}

Postoje mnoge druge konfiguracijske postavke, no ja vam preporučam da za početak postavite barem dvije:

S \verb+color.ui+ možete postaviti da ispis git naredbi bude obojan:

\begin{verbatim}
$ git config --global color.ui auto
\end{verbatim}

\verb+merge.tool+ određuje koji će se program koristiti u slučaju da \emph{konflikta} (o tome više kasnije). Ja koristim \verb+gvimdiff+:

\begin{verbatim}
$ git config --global merge.tool gvimdiff
\end{verbatim}

%\section*{}
%\addcontentsline{toc}{section}{}

