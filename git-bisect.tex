\tocChapter{$Bisect$}

\emph{Bisect} je git naredba koja se koristi kad tražimo izmjenu u kodu koja je uzrokovala grešku ($bug$) u programu.
Da bi mogli koristiti $bisect$ važno je da:

\begin{itemize}
    \item Imamo način kako utvrditi da li se bug manifestira u kodu kojeg trenutno imamo. Na primjer, $unit$ test ili shell skriptu koja provjerava postojanje buga.
    \item Znamo da je $bug$ nastao u nekom trenutku u povijesti projekta (tj. sigurni smo da se nije manifestirao u ranijim verzijama aplikacije).
\end{itemize}

Ukoliko su oba kriterija zadovoljena, moramo znati koji je točno raspon commitova u kojima tražimo $bug$.
Uzmimo na primjer da naš projekt ima povijest:

\input{graphs/bisect_1}

Znamo li da je $c$ commit u kojemu se $bug$ nije manifestirao, a zadnje stanje u našem $branch$u $k$ je pozicija gdje se bug manifestirao, onda znamo i da je bug nastao negdje u $commit$ovima između ta dva.

$Bisect$ nije ništa drugo nego binarno pretraživanje po povijesti projekta. 
U prvom koraku moramo gitu dati do znanja koji je $commit$ dobar (tj. u kojemu se $bug$ \textbf{nije manifestirao}), a koji je loš (tj. u kojemu se $bug$ \textbf{manifestira}).
I nakon toga slijedi niz iteracija pri čemu nas git prebacuje na neki $commit$ između zadnjeg dobrog i lošeg.
U svakom koraku se interval suzuje sve dok ne dođemo do mjesta kad je problem nastao.

Uzmimo, na primjer, da se trenutno nalazimo u grani u kojoj imamo bug.
Gitu dajemo do znanja da želimo započeti $bisect$ i da je trenutni $commit$ "\textbf{loš}":

\input{git_output/git_bisect_1}

Prebacujemo se na stanje za koje smo sigurni da se $bug$ nije manifestiralo:

\input{git_output/git_bisect_2}

Dajemo mu do znanja da je to "\textbf{dobar} $commit$"\footnote{Naravno, ako niste sigurni probajte otići još malo dalje u povijest sve dok ne dođete do $commit$a gdje se $bug$ sigurno ne manifestira}:

\input{git_output/git_bisect_3}

Git nas sad prebacuje na neki $commit$ između ta dva.
Nama je irelevantno koji je točno.
Jedino što trebamo je isprobati pojavljuje li se $bug$ u kodu koji je trenutno $checkout$an.

Na primjer, kod mene se $bug$ manifestirao, dakle pišem:

\input{git_output/git_bisect_4}

U sljedećem koraku sam na $commit$u gdje $bug$a nema:

\input{git_output/git_bisect_5}

\dots{}i dalje\dots

\input{git_output/git_bisect_6}

\dots{}sve dok u nekom trenutku ne dođemo do krivca:

\input{git_output/git_bisect_7}

Dakle, krivac je $commit$ \verb+b87db36d71038074a1c478c9f9a329d5c1685a02+.
Da bi točno pogledali što je tada promijenjeno, možemo:

\gitoutputcommand{git diff b87db36d7\textasciitilde{}1 b87db36d7}

Ili:

\gitoutputcommand{gitk b87db36d7}

Ukoliko u bilo kojem trenutku $bisect$ želimo prekinuti i vratiti se na mjesto ($commit$) gdje smo bili kad smo započeli, naredba je:

\gitoutputcommand{git bisect reset}

\tocSection{Digresija o atomarnim $commit$ovima}

$Bisect$ vam neće otkriti točan uzrok problema, samo će vamo točno reći koje su se izmijene dogodile kad je problem nastao, ali i to je često dovoljno kod traženja pravog uzroka. 

Nađe li vam $bisect$ da ste u tom $commit$u promijenili 5 linija koda tada barem jedna od tih linija koda nužno mora biti krivac.
Ako ste u tom $commit$u promijenili 100 linija koda među njima je krivac, ali lakše je bug tražiti u 5 nego u 100 izmijenjenih linija.

Radite li $commit$ove koji imaju tisuće promijenjenih linija, onda je puno teže naći uzrok.
No, to je ionako kriv pristup.
Ne bi nikad smjeli snimati više od jedne izmjene u koraku.
Svaki $commit$ bi uvijek trebao predstavljati jednu jedinu logičku cjelinu.

Na primjer, ako ste izmijenili dokumentaciju, ispravili nevezani bug i preformatirali kod u nekoj trećoj klasi -- to ne bi nikako smio biti \textbf{jedan} $commit$ nego \textbf{tri}.

Drugim riječima, treba raditi \textbf{atomarne $commit$ove}.

Na taj način će vam izmjene u kodu uvijek biti male, a i $bisect$ će nam s puno većom preciznošću moći pomoći kod traženja uzroka problema.
