\chapter*{Manje korištene naredbe}
\addcontentsline{toc}{chapter}{Manje korištene naredbe}

U ovom poglavlju ćemo proći neke rijeđe korištene naredbe gita.
Neke od njih ćete koristiti jako rijetko, a neke vjerojatno nikad.
Zato nije ni potrebno da ih detaljno razumijete, važno je samo da znate da one postoje. 
Ovdje ćemo ih samo nabrojati i generalno opisati čemu služe, a ako nekad zatrebaju -- lako ćete saznati njihovo detaljno korištenje s \verb+git help+.

\section*{Clean}
\addcontentsline{toc}{section}{Clean}

\section*{Bisect}
\addcontentsline{toc}{section}{Bisect}

\emph{Bisect} je pomoćna git naredba koja se koristi kad imamo bug u programu, a na znamo točno trenutak u povijesti repozitorija kad je nastao.

\TODO

\section*{Rev-list}
\addcontentsline{toc}{section}{Rev-list}

\TODO

\section*{Filter-branch}
\addcontentsline{toc}{section}{Filter-branch}

\TODO

\section*{Shortlog}
\addcontentsline{toc}{section}{Shortlog}

\TODO

\section*{Format-patch}
\addcontentsline{toc}{section}{Format-patch}

\TODO

\section*{Am}
\addcontentsline{toc}{section}{Am}

\TODO

\section*{Fsck}
\addcontentsline{toc}{section}{Fsck}

\TODO

\section*{Instaweb}
\addcontentsline{toc}{section}{Instaweb}

\TODO

\section*{Name-rev}
\addcontentsline{toc}{section}{Name-rev}

\TODO

\section*{Stash}
\addcontentsline{toc}{section}{Stash}

\TODO

\section*{Submodule}
\addcontentsline{toc}{section}{Submodule}

\TODO

%\section*{}
%\addcontentsline{toc}{section}{}
